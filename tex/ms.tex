\documentclass[modern]{aastex62}

% Load the corTeX style definitions
\input{cortex}

% Bibliography stuff
\bibliographystyle{aasjournal}

% Begin!
\begin{document}

% Title
\title{Doppler Imaging Fun}

% Author list
\author[0000-0002-0296-3826]{Rodrigo Luger}
\email{rluger@flatironinstitute.org}
\affil{Center~for~Computational~Astrophysics, Flatiron~Institute, New~York, NY}
%
\author{Megan Bedell}
\affil{Center~for~Computational~Astrophysics, Flatiron~Institute, New~York, NY}
%
\author{Probably David W. Hogg}
\affil{Center~for~Computational~Astrophysics, Flatiron~Institute, New~York, NY}

%
\section{Introduction}
%
Check out \citet{Luger2019} and \citet{Bedell2019} and stuff.

%
\section{The problem}
\label{sec:the_problem}
%
Let $I(\xi, x, y)$ be the Doppler-shifted intensity observed at log wavelength 
$\xi \equiv \ln\lambda$ at sky-projected Cartesian position $x, y$ on the 
surface of the star. We may express it as
%
\begin{align}
    \label{eq:I}
    I(\xi, x, y) &= I_0(\xi_0, x, y)
\end{align}
%
where $I_0$ is the spectrum at each point on the star in the surface rest
frame and 
%
\begin{align}
    \label{eq:xi0}
    \xi_0 &= \xi + \alpha(x, y)
\end{align}
%
is the log wavelength in the unshifted frame. The quantity $\alpha$ is
computed from the formula for the Doppler shift and is equal to
%
\begin{align}
    \label{eq:alpha}
    \alpha(x, y) 
        &=
        \frac{1}{2}\ln\left( 
            \frac{1 - \beta}{1 + \beta} 
        \right) 
        \nonumber \\
        &\approx
        \ln(1 - \beta)
\end{align}
%
where $\beta = v(x, y) / c$ is the ratio of the 
radial velocity at a point on the surface of the star to the speed of light
and the second line is valid when $\beta \ll 1$ (i.e., the non-relativistic
Doppler shift). Throughout this paper, we will assume the non-relativistic 
limit, as it greatly simplifies the math.
In keeping with the literature, we take positive values of $v$ to mean 
redshifts.

A common approach to computing the Doppler-shifted spectrum is to
evaluate the spectrum at the rest frame wavelength (Equation~\ref{eq:xi0})
and interpolate back to the grid in $\xi$. This is practical when
computing the spectrum at a single \emph{point} on the surface, but not
ideal when one is interested in the \emph{integral} over the entire
surface of the star $\mathcal{S}$, which is typically all we can observe:
%
\begin{align}
    \label{eq:S}
    S(\xi) 
        &\equiv
        \iint\limits_{\mathcal{S}(x, y)}
                I(\xi, x, y)
        \mathrm{d}{\mathcal{S}(x, y)}
        \quad .
\end{align}
%
The difficulty in solving Equation~(\ref{eq:S}) stems from the fact
that $I(\xi, x, y)$ is difficult to write down in closed form, given
the non-linearity of the Doppler shift.
The standard approach to solving this integral is therefore
to discretize the surface of the star with a fine grid, evaluate the
Doppler-shifted spectrum in each cell, and sum over the spatial axes
to approximate the integral. Depending on the resolution of the grid,
this is either numerically inaccurate or computationally inefficient 
(and often both).

\section{The Solution}

In this section, we consider two alternative methods to solving
Equation~(\ref{eq:S}) to improve the speed and accuracy of the
computation, both of which rely on the Fourier transform of the
spectrum.
Recall the definition of the Fourier transform of a function $f(\xi)$,
%
\begin{align}
    \label{eq:fourier:FT}
    \mathcal{F}\Big[f(\xi)\Big](k) 
    &\equiv
    \int_{-\infty}^\infty
        f(\xi)
        \mathrm{e}^{-2\pi i k \xi} \mathrm{d}\xi
    \nonumber \\
    &=
    \hat{f}(k)
    \quad ,
\end{align}
%
and the corresponding inverse Fourier transform,
%
\begin{align}
    \label{eq:fourier:IFT}
    \mathcal{F}^{-1}\Big[\hat{f}(k)\Big](\xi) 
    &\equiv
    \int_{-\infty}^\infty
        \hat{f}(k)
        \mathrm{e}^{2\pi i k \xi} \mathrm{d}k
    \nonumber \\
    &=
    f(\xi)
    \quad .
\end{align}
%
Consider what happens when we take the
Fourier transform of the spectrum $I(\xi, x, y)$ to obtain
a function $\hat{I}(k, x, y)$, where we refer to $k$ as a ``wavenumber''. 
A useful property of the
Fourier transform is that a translation of $I(\xi, x, y)$ by
an amount $\Delta\xi$ corresponds to a linear scaling
of $\hat{I}(k, x, y)$ by an amount $\mathrm{e}^{-2\pi i k \Delta\xi}$
\citep[e.g.,][]{Schoenstadt2006}.
Specifically, if the Doppler-shifted spectrum is given by
%
\begin{align}
    I(\xi, x, y) &= I_0\big(\xi + \alpha(x, y)\big), x, y) \quad,
\end{align}
%
its Fourier transform is equal to 
%
\begin{align}
    \label{eq:fourier:translation}
    \hat{I}(\xi, x, y) 
    &= 
    \mathrm{e}^{2\pi i k \alpha(x, y)}\hat{I_0}(k, x, y)
    \quad ,
\end{align}
%
where $\hat{I_0}(k, x, y) \equiv \mathcal{F}\left[ I_0(\xi, x, y) \right]$.
%
To obtain the Fourier transform of the observed spectrum, we
expand $\hat{I_0}(k, x, y)$ in terms of spherical harmonics,
%
\begin{align}
    \label{eq:fourier:I0}
    \hat{I_0}(k, x, y) 
        &=
        \sum_{l=0}^\infty\sum_{m=-l}^{l} \hat{a}_{lm}(k) Y_{lm}(x, y)
\end{align}
%
then take the integral of 
Equation~(\ref{eq:fourier:translation})
over the visible disk $\mathcal{S}$ of the star:
%
\begin{align}
    \label{eq:fourier:SYlm}
    \hat{S}(k) 
    &=
        \sum_{l=0}^\infty\sum_{m=-l}^{l}
        \hat{a}_{lm}(k)
        \iint\limits_{\mathcal{S}(x, y)}
        Y_{lm}(x, y)
        \mathrm{e}^{2\pi i k \alpha(x, y)}
        \mathrm{d}\mathcal{S}(x, y)
    \nonumber \\
    &=
        \iint\limits_{\mathcal{S}(x, y)}
        \tilde{\bvec{y}}(x, y)
        \mathrm{e}^{2\pi i k \alpha(x, y)}
        \mathrm{d}\mathcal{S}(x, y)
        \cdot
        \bvec{\hat{a}}(k)
    \quad ,
\end{align}
%
where
%
\begin{align}
    \bvec{\hat{a}}(k) \equiv
\Big( 
    \hat{a}_{0,0}(k) \quad\quad\quad\quad\quad\quad 
    \hat{a}_{1,-1}(k) \quad\quad\quad\quad\quad\quad 
    \hat{a}_{1,0}(k) \quad\quad\quad\quad\quad\quad
    \hat{a}_{1,1}(k) \quad\quad\quad\quad\quad\quad 
    ... 
\Big)^\top
\end{align}
%
is the column vector of spherical harmonic coefficients at wavenumber $k$, and
%
\begin{align}
    \tilde{\bvec{y}}(x, y) \equiv 
\Big( 
    Y_{0,0}(x, y) \quad\quad\quad\quad\quad\quad 
    Y_{1,-1}(x, y) \quad\quad\quad\quad\quad\quad 
    Y_{1,0}(x, y) \quad\quad\quad\quad\quad\quad 
    Y_{1,1}(x, y) \quad\quad\quad\quad\quad\quad 
    ... 
\Big)
\end{align}
%
is the corresponding row vector of spherical harmonics at a point $(x, y)$ 
on the sphere. In order to tackle the integrals in 
Equation~(\ref{eq:fourier:SYlm}), let us change bases from spherical
harmonics to polynomials on the sphere:
%
\begin{align}
    \label{eq:fourier:S}
    \hat{S}(k) 
    &=
        \iint\limits_{\mathcal{S}(x, y)}
        \tilde{\bvec{p}}(x, y)
        \mathrm{e}^{2\pi i k \alpha(x, y)}
        \mathrm{d}\mathcal{S}(x, y)
        \cdot
        \bvec{A_1}
        \bvec{\hat{a}}(k)
    \quad .
\end{align}
%
where
%
\begin{align}
    \tilde{\bvec{p}}(x, y) \equiv 
\Big( 
    1 \quad\quad\quad\quad\quad\quad 
    x \quad\quad\quad\quad\quad\quad 
    z \quad\quad\quad\quad\quad\quad 
    y \quad\quad\quad\quad\quad\quad 
    x^2 \quad\quad\quad\quad\quad\quad 
    xz \quad\quad\quad\quad\quad\quad 
    xy \quad\quad\quad\quad\quad\quad
    yz \quad\quad\quad\quad\quad\quad 
    y^2 \quad\quad\quad\quad\quad\quad
    ... 
\Big)
\end{align}
%
is the polynomial basis \citep[Equation 7 in][]{Luger2019}
and $\bvec{A_1}$ is the change of basis matrix from spherical harmonics
to polynomials 
\citep[Equation B11 in][]{Luger2019}.

At this point, we have two options: we can remain in Fourier space and
attempt to solve the surface integral in Equation~(\ref{eq:fourier:S})
directly, or we can take the inverse Fourier transform of  
Equation~(\ref{eq:fourier:S}) and attempt to solve the integral in
log wavelength space. As we will see, there are advantages (and
disadvantages) to both methods. We do the former in \S\ref{sec:analytic}
in the latter in \S\ref{sec:convolve}.

\subsection{Analytic solutions}
\label{sec:analytic}
In the case that the integration area $\mathcal{S}$ is the entire surface of
the star, the integrals in Equation~(\ref{eq:fourier:S}) are
\emph{analytic}. \xxx{Show here how the result can be computed from the 
hypergeometric function $_3F_2.$}

\subsection{Doppler shifts as a convolution}
\label{sec:convolve}

If we take the inverse Fourier transform of Equation~(\ref{eq:fourier:S}),
we arrive at an expression for the observed spectrum:
%
\begin{align}
    \label{eq:convolve:SUgly}
    S(\xi) 
    &=
    \mathcal{F}^{-1} \Bigg[\,\,\,
        \iint\limits_{\mathcal{S}(x, y)}
        \tilde{\bvec{p}}(x, y)
        \mathrm{e}^{2\pi i k \alpha(x, y)}
        \mathrm{d}\mathcal{S}(x, y)
        \cdot
        \bvec{A_1}
        \bvec{\hat{a}}(k)
    \Bigg] \nonumber \\
    &=
    \mathcal{F}^{-1} \Bigg[\,\,\,
        \iint\limits_{\mathcal{S}(x, y)}
        \tilde{\bvec{p}}(x, y)
        \mathrm{e}^{2\pi i k \alpha(x, y)}
        \mathrm{d}\mathcal{S}(x, y)
    \Bigg]
    \circ
    \bvec{A_1}
    \bvec{a}(\xi)
    \quad ,
\end{align}
%
where the second line follows from the convolution theorem, with
$\circ$ denoting the convolution operator over $\xi$ and 
$\bvec{a}(\xi) \equiv \mathcal{F}^{-1}\left[ \bvec{\hat{a}}(k) \right]$.
Since only the exponential term in the integrand depends on $k$, we may 
further simplify this expression:
%
\begin{align}
    \label{eq:convolve:SUgly}
    S(\xi) 
    &=    
    \iint\limits_{\mathcal{S}(x, y)}
    \tilde{\bvec{p}}(x, y)
    \mathcal{F}^{-1} \Big[
        \mathrm{e}^{2\pi i k \alpha(x, y)}
    \Big]
    \mathrm{d}\mathcal{S}(x, y)
    \circ
    \bvec{A_1}
    \bvec{a}(\xi)
    \nonumber \\
    &=    
    \iint\limits_{\mathcal{S}(x, y)}
    \tilde{\bvec{p}}(x, y)
    \delta\big(\xi - \alpha(x, y)\big)
    \mathrm{d}\mathcal{S}(x, y)
    \circ
    \bvec{A_1}
    \bvec{a}(\xi)
    \quad,
\end{align}
%
where $\delta$ is the Dirac delta function. Since the integrand in the
equation above is zero everywhere except where the argument of the
$\delta$ function is zero, the equation may be written as a line integral 
along some curve $\mathcal{C}(x, y)$ along which $\xi = \alpha(x, y)$:
%
\begin{align}
    \label{eq:convolve:SUgly}
    S(\xi) 
    &=    
    \int\limits_{\mathcal{C}(x, y)}
    \tilde{\bvec{p}}(x, y)
    \mathrm{d}\mathcal{C}(x, y)
    \circ
    \bvec{A_1}
    \bvec{a}(\xi)
    \quad.
\end{align}
%


\appendix

\section{Taylor Expanding the Doppler operator \\ (and why this is a bad idea)}
\label{sec:taylor}
%
An alternative way to tackle the problem is to linearize the Doppler 
operator via a Taylor expansion in $\alpha$ about $\alpha=0$:
%
\begin{align}
    \label{eq:taylor:I}
    I(\xi, x, y) 
        &=
        I_0(\xi_0, x, y) \Bigg|_{\alpha=0}
        + 
        \frac{\mathrm{d}I_0(\xi_0, x, y)}{\mathrm{d}\alpha} \Bigg|_{\alpha=0} 
            \Delta\alpha(x, y)
        + 
        \frac{1}{2}\frac{\mathrm{d}^2I_0(\xi_0, x, y)}{\mathrm{d}\alpha^2} 
            \Bigg|_{\alpha=0} \Delta\alpha(x, y)^2
        +
        ... 
\end{align}
%
Since the dependence of Equation~(\ref{eq:xi0}) on $\alpha$ is trivial,
the derivatives of the spectrum $I_0(\xi_0)$ with respect to
$\alpha$ are simply
%
\begin{align}
    \frac{\mathrm{d}^nI_0(\xi_0, x, y)}{\mathrm{d}\alpha^n} &=
    \dfrac{\mathrm{d}^nI_0(\xi_0, x, y)}{\mathrm{d}\xi_0^n} \nonumber\\ &\equiv
    I_0^{(n)}(\xi_0, x, y)
    \quad.
\end{align}
%
Given this result, and noting that $\xi_0 = \xi$ when $\alpha = 0$,
we may re-write Equation~(\ref{eq:taylor:I}) as
%
\begin{proof}{Taylor}
    \label{eq:taylor:ISum}
    I(\xi, x, y) 
        &=
        \sum_{n=0}^\infty
            \frac{I_{0}^{(n)}(\xi, x, y)}{n!}
            \Delta\alpha(x, y)^n
        \quad ,
\end{proof}
%
The utility of this expression is that the velocity dependence of the spectrum
is now entirely encoded in the terms $\Delta\alpha(x, y)^n$, which are
\emph{independent of wavelength}. We can further decouple the spatial
dependence from the spectral dependence by expressing the intensity field
as an expansion over spherical harmonics $Y_{lm}(x, y)$ in the sky-projected
coordinates:
%
\begin{align}
    I_0(\xi, x, y) = \sum_{l=0}^\infty\sum_{m=-l}^{l} a_{lm}(\xi) Y_{lm}(x, y)
    \quad .
\end{align}
%
Equation~(\ref{eq:taylor:ISum}) now reads
%
\begin{align}
    \label{eq:taylor:IYlm}
    I(\xi, x, y) 
        &=
        \sum_{n=0}^\infty
            \sum_{l=0}^\infty\sum_{m=-l}^{l}
                \frac{a_{lm}^{(n)}(\xi)}{n!}
                Y_{lm}(x, y)\Delta\alpha(x, y)^n
            \quad .
\end{align}
%
Finally, integrating this equation over the visible disk of 
the star, we arrive at an equation for the observed spectrum:
%
\begin{align}
    \label{eq:taylor:S}
    S(\xi) 
        &=
        \sum_{n=0}^\infty
            \sum_{l=0}^\infty\sum_{m=-l}^{l}
                \frac{a_{lm}^{(n)}(\xi)}{n!}
                \iint\limits_{\mathcal{S}(x, y)}
                Y_{lm}(x, y)\Delta\alpha(x, y)^n
                \mathrm{d}{\mathcal{S}(x, y)}
            \quad .
\end{align}
%
By expanding the spectrum in both the spectral and spatial dimensions, we
have effectively decoupled the two. However, Equation~(\ref{eq:taylor:S}) 
is impractical for three reasons. First, the surface integrals can be very
difficult to solve (although we will show later how these integrals may be
approximated analytically). Second, it requires knowledge of high order 
derivatives of the spectrum, which may not always be easy or convenient to 
compute. Third, and most important, the series expansion in $n$ is typically 
extremely slow to converge and is therefore only practically useful in the 
limit that the Doppler shift is much smaller than the typical width of a 
spectral line; see the notebook link next to Equation~(\ref{eq:taylor:ISum}) 
for an example.

% Bibliography
\bibliography{bib}

\end{document}
