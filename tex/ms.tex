\documentclass[modern]{aastex62}

% Load the corTeX style definitions
\input{cortex}

% Bibliography stuff
\bibliographystyle{aasjournal}

% Begin!
\begin{document}

% Title
\title{Doppler Imaging Fun}

% Author list
\author[0000-0002-0296-3826]{Rodrigo Luger}
\email{rluger@flatironinstitute.org}
\affil{Center~for~Computational~Astrophysics, Flatiron~Institute, New~York, NY}
%
\author{Megan Bedell}
\affil{Center~for~Computational~Astrophysics, Flatiron~Institute, New~York, NY}
%
\author{Probably David W. Hogg}
\affil{Center~for~Computational~Astrophysics, Flatiron~Institute, New~York, NY}

%
\section{Introduction}
Check out \citet{Luger2019} and \citet{Bedell2019} and stuff.

%
\section{The equation}
\label{sec:the_equation}

In the most general form, the Doppler-shifted intensity observed at wavelength $\lambda$ at
position $x, y$ on the surface of the star at time $t$ is
%
\begin{align}
    I(\lambda, \beta, x, y, t) &=
        I\big(\lambda, 0, x, y, t\big) \nonumber \\
        &+ \frac{\mathrm{d}I(\lambda, \beta, x, y, t)}{\mathrm{d}\beta} \Big|_{\beta=0} \Delta\beta(x, y) \nonumber \\
        &+ \frac{\mathrm{d}^2I(\lambda, \beta, x, y, t)}{\mathrm{d}\beta^2} \Big|_{\beta=0} \Delta\beta^2(x, y) \nonumber \\
        &+ ...
\end{align}
%
where $\beta \equiv \frac{v}{c}$ is the relativistic parameter for a radial velocity $v$ on the surface.

% 
\section{Differentiating the spectrum}
\label{sec:derivatives}
The derivatives of the spectrum $I(\lambda)$ with respect to the relativistic parameter
$\beta$ are found by application of Fa\`a di Bruno's formula for taking high
order derivatives of the chain rule:
%
\begin{proof}{faa_di_bruno}
    \label{eq:dIdbeta}
    \frac{\mathrm{d}^n I(\lambda, \beta)}{\mathrm{d}\beta^n} \Big|_{\beta=0} &=
    \sum_{k=1}^n \frac{\mathrm{d}^k I(\lambda_0)}{\mathrm{d}\lambda_0^k} \Big|_{\lambda_0=\lambda} \lambda^k P_{nk}
\end{proof}
%
where
%
\begin{proof}{faa_di_bruno}
    \label{eq:Pnk}
    P_{nk} \equiv B_{n, k}\Bigg( \Big\{(-1)^j j! \Big\}_{j=1}^{n - k + 1} \Bigg)
\end{proof}
%
and $B_{n, k}$ is the incomplete Bell polynomial. The quantity
$\frac{\mathrm{d}^k I(\lambda_0)}{\mathrm{d}\lambda_0^k} \Big|_{\lambda_0=\lambda}$
is just the $k^\mathrm{th}$ derivative of the spectrum with respect to wavelength in the
rest frame, and must either be inferred from the data or computed numerically
from the spectrum.


% Bibliography
\pagebreak
\bibliography{bib}

\end{document}
