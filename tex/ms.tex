\documentclass[modern]{aastex62}

% Load the corTeX style definitions
% All the packages
\usepackage{url}
\usepackage{amsmath}
\usepackage{mathtools}
\usepackage{amssymb}
\usepackage{natbib}
\usepackage{graphicx}
\usepackage{calc}
\usepackage{etoolbox}
\usepackage{xspace}
\usepackage[T1]{fontenc} % https://tex.stackexchange.com/a/166791
\usepackage{textcomp}
\usepackage{ifxetex}
\ifxetex
\usepackage{fontspec}
\defaultfontfeatures{Extension = .otf}
\fi
\usepackage{fontawesome}
\usepackage{listings}
\usepackage{nicefrac}
\usepackage[bb=boondox]{mathalfa}


% Shorthand for this paper
\newcommand{\Python}{\textsf{Python}\xspace}
\newcommand{\cpp}{\textsf{C}++\xspace}
\newcommand{\bvec}[1]{{\ensuremath{\mathbf{#1}}}}
\newcommand{\xxx}[1]{{\color{red}#1}}
\DeclarePairedDelimiter\floor{\lfloor}{\rfloor}
\DeclarePairedDelimiter\ceil{\lceil}{\rceil}
\newcommand{\imag}{{\ensuremath{\mathbb{i}}}}

% References to text content
\newcommand{\documentname}{\textsl{article}}
\newcommand{\figureref}[1]{\ref{fig:#1}}
\newcommand{\Figure}[1]{Figure~\figureref{#1}}
\newcommand{\figurelabel}[1]{\label{fig:#1}}
\renewcommand{\eqref}[1]{\ref{eq:#1}}
\newcommand{\Eq}[1]{Equation~(\eqref{#1})}
\newcommand{\eq}[1]{\Eq{#1}}
\newcommand{\eqalt}[1]{Equation~\eqref{#1}}

% Add code, proof, and animation hyperlinks
\definecolor{linkcolor}{rgb}{0.1216,0.4667,0.7059}
\newcommand{\codeicon}{{\color{linkcolor}\faFileCodeO}}
\newcommand{\prooficon}{{\color{linkcolor}\faPencilSquareO}}
% !TeX root = ./ms.tex
\newcommand{\codelink}[1]{\href{https://github.com/user/repo/blob/076a0d29804b1875a480b0fd74a7ea6738368263/tex/figures/#1.py}{\codeicon}\,\,}
\newcommand{\animlink}[1]{\href{https://github.com/user/repo/blob/076a0d29804b1875a480b0fd74a7ea6738368263/tex/figures/#1.gif}{\animicon}\,\,}
\newcommand{\prooflink}[1]{\href{https://github.com/user/repo/blob/076a0d29804b1875a480b0fd74a7ea6738368263/tex/proofs/#1.ipynb}{\raisebox{-0.1em}{\prooficon}}}
\newcommand{\cilink}[1]{\href{https://dev.azure.com/user/repo/_build}{#1}}


% Define a proof environment for open source equation proofs
\newtagform{eqtag}[]{(}{)}
\newcommand{\currentlabel}{None}
\newenvironment{proof}[1]{%
\ifstrempty{#1}{%
\renewtagform{eqtag}[]{\raisebox{-0.1em}{{\color{red}\faPencilSquareO}}\,(}{)}%
}{%
\renewtagform{eqtag}[]{\prooflink{#1}\,(}{)}%
}%
\usetagform{eqtag}%
\renewcommand{\currentlabel}{#1}
\align%
}{%
\endalign%
\renewtagform{eqtag}[]{(}{)}%
\usetagform{eqtag}%
\message{<<<\currentlabel: \theequation>>>}%
}

% Define the `oscaption` command for open source figure captions
\newcommand{\oscaption}[2]{\caption{#2 \codelink{#1}}}

% Code examples
\definecolor{codegreen}{rgb}{0,0.6,0}
\definecolor{codegray}{rgb}{0.5,0.5,0.5}
\definecolor{codepurple}{rgb}{0.58,0,0.82}
\definecolor{backcolour}{rgb}{0.95,0.95,0.95}
\lstdefinestyle{mystyle}{
    backgroundcolor=\color{backcolour},
    commentstyle=\color{codegreen},
    keywordstyle=\color{magenta},
    numberstyle=\tiny\color{codegray},
    stringstyle=\color{codepurple},
    basicstyle=\small\ttfamily,
    breakatwhitespace=false,
    breaklines=true,
    captionpos=b,
    keepspaces=true,
    numbers=left,
    numbersep=5pt,
    showspaces=false,
    showstringspaces=false,
    showtabs=false,
    tabsize=2,
    aboveskip=1em,
    belowskip=1em,
    keywords=[2]{map},
    keywordstyle=[2]{\color{black!80!black}},
    upquote=true
}
\lstset{style=mystyle}

% Typography obsessions
\setlength{\parindent}{3.0ex}
\renewcommand\quad{\hskip\fontdimen3\font}


% Bibliography stuff
\bibliographystyle{aasjournal}

% Begin!
\begin{document}

% Title
\title{Doppler Imaging Fun}

% Author list
\author[0000-0002-0296-3826]{Rodrigo Luger}
\email{rluger@flatironinstitute.org}
\affil{Center~for~Computational~Astrophysics, Flatiron~Institute, New~York, NY}
%
\author{Megan Bedell}
\affil{Center~for~Computational~Astrophysics, Flatiron~Institute, New~York, NY}

%
\section{Introduction}
Check out \citet{Luger2019} and \citet{Bedell2019} and stuff.

%
\section{The equation}
\label{sec:the_equation}

In the most general form, the Doppler-shifted intensity observed at wavelength $\lambda$ at
position $x, y$ on the surface of the star at time $t$ is
%
\begin{align}
    I(\lambda, \beta, x, y, t) &=
        I\big(\lambda, 0, x, y, t\big) \nonumber \\
        &+ \frac{\mathrm{d}I(\lambda, \beta, x, y, t)}{\mathrm{d}\beta} \Big|_{\beta=0} \Delta\beta(x, y) \nonumber \\
        &+ \frac{\mathrm{d}^2I(\lambda, \beta, x, y, t)}{\mathrm{d}\beta^2} \Big|_{\beta=0} \Delta\beta^2(x, y) \nonumber \\
        &+ ...
\end{align}
%
where $\beta \equiv \frac{v}{c}$ is the relativistic parameter for a radial velocity $v$ on the surface.

% 
\section{Differentiating the spectrum}
\label{sec:derivatives}
The derivatives of the spectrum $I(\lambda)$ with respect to the relativistic parameter
$\beta$ are found by application of Fa\`a di Bruno's formula for taking high
order derivatives of the chain rule:
%
\begin{proof}{faa_di_bruno}
    \label{eq:dIdbeta}
    \frac{\mathrm{d}^n I(\lambda, \beta)}{\mathrm{d}\beta^n} \Big|_{\beta=0} &=
    \sum_{k=1}^n \frac{\mathrm{d}^k I(\lambda_0)}{\mathrm{d}\lambda_0^k} \Big|_{\lambda_0=\lambda} \lambda^k P_{nk}
\end{proof}
%
where
%
\begin{proof}{faa_di_bruno}
    \label{eq:Pnk}
    P_{nk} \equiv B_{n, k}\Bigg( \Big\{(-1)^j j! \Big\}_{j=1}^{n - k + 1} \Bigg)
\end{proof}
%
and $B_{n, k}$ is the incomplete Bell polynomial. The quantity
$\frac{\mathrm{d}^k I(\lambda_0)}{\mathrm{d}\lambda_0^k} \Big|_{\lambda_0=\lambda}$
is just the $k^\mathrm{th}$ derivative of the spectrum with respect to wavelength in the
rest frame, and must either be inferred from the data or computed numerically
from the spectrum.


% Bibliography
\pagebreak
\bibliography{bib}

\end{document}
