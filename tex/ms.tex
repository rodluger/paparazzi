\documentclass[modern]{aastex62}

% Load the corTeX style definitions
% All the packages
\usepackage{url}
\usepackage{amsmath}
\usepackage{mathtools}
\usepackage{amssymb}
\usepackage{natbib}
\usepackage{graphicx}
\usepackage{calc}
\usepackage{etoolbox}
\usepackage{xspace}
\usepackage[T1]{fontenc} % https://tex.stackexchange.com/a/166791
\usepackage{textcomp}
\usepackage{ifxetex}
\ifxetex
\usepackage{fontspec}
\defaultfontfeatures{Extension = .otf}
\fi
\usepackage{fontawesome}
\usepackage{listings}
\usepackage{nicefrac}
\usepackage[bb=boondox]{mathalfa}


% Shorthand for this paper
\newcommand{\Python}{\textsf{Python}\xspace}
\newcommand{\cpp}{\textsf{C}++\xspace}
\newcommand{\bvec}[1]{{\ensuremath{\mathbf{#1}}}}
\newcommand{\xxx}[1]{{\color{red}#1}}
\DeclarePairedDelimiter\floor{\lfloor}{\rfloor}
\DeclarePairedDelimiter\ceil{\lceil}{\rceil}
\newcommand{\imag}{{\ensuremath{\mathbb{i}}}}

% References to text content
\newcommand{\documentname}{\textsl{article}}
\newcommand{\figureref}[1]{\ref{fig:#1}}
\newcommand{\Figure}[1]{Figure~\figureref{#1}}
\newcommand{\figurelabel}[1]{\label{fig:#1}}
\renewcommand{\eqref}[1]{\ref{eq:#1}}
\newcommand{\Eq}[1]{Equation~(\eqref{#1})}
\newcommand{\eq}[1]{\Eq{#1}}
\newcommand{\eqalt}[1]{Equation~\eqref{#1}}

% Add code, proof, and animation hyperlinks
\definecolor{linkcolor}{rgb}{0.1216,0.4667,0.7059}
\newcommand{\codeicon}{{\color{linkcolor}\faFileCodeO}}
\newcommand{\prooficon}{{\color{linkcolor}\faPencilSquareO}}
% !TeX root = ./ms.tex
\newcommand{\codelink}[1]{\href{https://github.com/user/repo/blob/076a0d29804b1875a480b0fd74a7ea6738368263/tex/figures/#1.py}{\codeicon}\,\,}
\newcommand{\animlink}[1]{\href{https://github.com/user/repo/blob/076a0d29804b1875a480b0fd74a7ea6738368263/tex/figures/#1.gif}{\animicon}\,\,}
\newcommand{\prooflink}[1]{\href{https://github.com/user/repo/blob/076a0d29804b1875a480b0fd74a7ea6738368263/tex/proofs/#1.ipynb}{\raisebox{-0.1em}{\prooficon}}}
\newcommand{\cilink}[1]{\href{https://dev.azure.com/user/repo/_build}{#1}}


% Define a proof environment for open source equation proofs
\newtagform{eqtag}[]{(}{)}
\newcommand{\currentlabel}{None}
\newenvironment{proof}[1]{%
\ifstrempty{#1}{%
\renewtagform{eqtag}[]{\raisebox{-0.1em}{{\color{red}\faPencilSquareO}}\,(}{)}%
}{%
\renewtagform{eqtag}[]{\prooflink{#1}\,(}{)}%
}%
\usetagform{eqtag}%
\renewcommand{\currentlabel}{#1}
\align%
}{%
\endalign%
\renewtagform{eqtag}[]{(}{)}%
\usetagform{eqtag}%
\message{<<<\currentlabel: \theequation>>>}%
}

% Define the `oscaption` command for open source figure captions
\newcommand{\oscaption}[2]{\caption{#2 \codelink{#1}}}

% Code examples
\definecolor{codegreen}{rgb}{0,0.6,0}
\definecolor{codegray}{rgb}{0.5,0.5,0.5}
\definecolor{codepurple}{rgb}{0.58,0,0.82}
\definecolor{backcolour}{rgb}{0.95,0.95,0.95}
\lstdefinestyle{mystyle}{
    backgroundcolor=\color{backcolour},
    commentstyle=\color{codegreen},
    keywordstyle=\color{magenta},
    numberstyle=\tiny\color{codegray},
    stringstyle=\color{codepurple},
    basicstyle=\small\ttfamily,
    breakatwhitespace=false,
    breaklines=true,
    captionpos=b,
    keepspaces=true,
    numbers=left,
    numbersep=5pt,
    showspaces=false,
    showstringspaces=false,
    showtabs=false,
    tabsize=2,
    aboveskip=1em,
    belowskip=1em,
    keywords=[2]{map},
    keywordstyle=[2]{\color{black!80!black}},
    upquote=true
}
\lstset{style=mystyle}

% Typography obsessions
\setlength{\parindent}{3.0ex}
\renewcommand\quad{\hskip\fontdimen3\font}


% Bibliography stuff
\bibliographystyle{aasjournal}

% Begin!
\begin{document}

% Title
\title{Doppler Imaging Fun}

% Author list
\author[0000-0002-0296-3826]{Rodrigo Luger}
\email{rluger@flatironinstitute.org}
\affil{Center~for~Computational~Astrophysics, Flatiron~Institute, New~York, NY}
%
\author{Megan Bedell}
\affil{Center~for~Computational~Astrophysics, Flatiron~Institute, New~York, NY}
%
\author{Probably David W. Hogg}
\affil{Center~for~Computational~Astrophysics, Flatiron~Institute, New~York, NY}

%
\section{Introduction}
Check out \citet{Luger2019} and \citet{Bedell2019} and stuff.

%
\section{Uniform surfaces}
\label{sec:uniform surfaces}

Let $I(\xi, x, y)$ be the Doppler-shifted intensity observed at log wavelength 
$\xi \equiv \ln\lambda$ at sky-projected Cartesian position $x, y$ on the 
surface of the star. We may express it as
%
\begin{align}
    \label{eq:IntensityUnif}
    I(\xi, x, y) &= I_0(\xi_0)
\end{align}
%
where $I_0$ is the (spatially uniform) spectrum in the original, unshifted 
frame and 
%
\begin{align}
    \label{eq:xi0}
    \xi_0 &= \xi - \alpha(x, y)
\end{align}
%
is the log wavelength in the unshifted frame. The quantity $\alpha$ is
computed from the formula for the Doppler shift and is equal to
%
\begin{align}
    \alpha(x, y) 
        &=
        \frac{1}{2}\ln\left( 
            \frac{1 - \beta}{1 + \beta} 
        \right) \nonumber \\
        &\approx
        \ln(1 - \beta)
\end{align}
%
where $\beta = v(x, y) / c$ is the ratio of the 
radial velocity at a point on 
the surface of the star to the speed of light. We assume the
$z$-axis points toward the observer, so positive values of $v$ correspond
to blueshifts.

A common approach to computing the Doppler-shifted spectrum is to
evaluate the spectrum at the rest frame wavelength (Equation~\ref{eq:xi0})
and interpolate back to the grid in $\xi$. This is practical when
computing the spectrum at a single \emph{point} on the surface, but not
ideal when one is interested in the \emph{integral} over the entire
surface, which is typically what we can observe. The standard approach is
to discretize the surface of the star with a fine grid, evaluate the
Doppler-shifted spectrum in each cell, and sum over the spatial axes
to approximate the integral. Depending on the resolution of the grid,
this is either numerically inaccurate or computationally inefficient 
(and often both).

\subsection{Linearizing the Doppler operator}
\label{sec:taylor}

One way to improve upon this could be to linearize the Doppler operator via
a Taylor expansion in $\alpha$ about $\alpha=0$:
%
\begin{align}
    \label{eq:TaylorUnifExplicit}
    I(\xi, x, y) 
        &=
        I_0(\xi_0) \Bigg|_{\alpha=0}
        + 
        \frac{\mathrm{d}I_0(\xi_0)}{\mathrm{d}\alpha} \Bigg|_{\alpha=0} 
            \Delta\alpha(x, y)
        + 
        \frac{1}{2}\frac{\mathrm{d}^2I_0(\xi_0)}{\mathrm{d}\alpha^2} 
            \Bigg|_{\alpha=0} \Delta\alpha(x, y)^2
        +
        ... 
\end{align}
%
The derivatives of the spectrum $I_0(\xi_0)$ with respect to
$\alpha$ are computed by repeated application of the chain rule:
%
\begin{align}
    \frac{\mathrm{d}^nI_0(\xi_0)}{\mathrm{d}\alpha^n} &=
    (-1)^n\dfrac{\mathrm{d}^nI_0(\xi_0)}{\mathrm{d}\xi_0^n}
\end{align}
%
Given this result, and noting that $\xi_0 = \xi$ when $\alpha = 0$,
we may re-write Equation~(\ref{eq:TaylorUnifExplicit}) as
%
\begin{proof}{Taylor}
    \label{eq:TaylorUnifSum}
    I(\xi, x, y) 
        &=
        I_0(\xi)
        +
        \sum_{n=1}^\infty
            \frac{(-1)^n}{n!}
            \frac{\mathrm{d}^nI_0(\xi)}{\mathrm{d}\xi^n}
            \Delta\alpha(x, y)^n
\end{proof}
%
The utility of this expression is that the spatial dependence of the spectrum
is now entirely encoded in the terms $\Delta\alpha(x, y)^n$, which is
\emph{independent of wavelength}. The surface integral of 
Equation~(\ref{eq:TaylorUnifSum}) is therefore far easier to compute, since
it is the integral of a \emph{scalar} function that could even be approximated
analytically for some choice of the velocity field $\alpha(x, y)$.

In practice, however, Equation~(\ref{eq:TaylorUnifSum}) is impractical for
two reasons. First, it requires knowledge of high order derivatives of the 
spectrum, which may not always be easy or convenient to compute. Second, and
more important, the series expansion is typically extremely slow to converge
and is therefore only practically useful in the limit that the Doppler shift
is much smaller than the typical width of a spectral line; see the notebook
link next to Equation~(\ref{eq:TaylorUnifSum}) for an example.

\subsection{Taking the Fourier transform}
\label{sec:fourier}

A much better approach is to consider what happens when we take the
Fourier transform of the spectrum $I(\xi, x, y)$, defined as
%
\begin{align}
    \label{eq:FT}
    \hat{I}(k, x, y)
    \equiv
    \mathcal{F}\Big[I(\xi, x, y)\Big](k) 
    =
    \int_{-\infty}^\infty
        I(\xi, x, y)
        \mathrm{e}^{-2\pi i k \xi} \mathrm{d}\xi
\end{align}
%
where $k$ is the transformed independent variable in units 
of inverse log wavelength. The utility of this transform is
that it effectively linearizes the Doppler operator, since
the Fourier transform of the translation of a function is
simply a linear scaling of the Fourier transform of the
original function. Specifically, if the Doppler-shifted spectrum 
is given by
%
\begin{align}
    I(\xi, x, y) &= I_0\big(\xi - \alpha(x, y)\big) \quad,
\end{align}
%
its Fourier transform is equal to
%
\begin{align}
    \hat{I}(\xi, x, y) &= \mathrm{e}^{-2\pi i k \alpha(x, y)}\hat{I_0}(k)
\end{align}
%

% Bibliography
\bibliography{bib}

\end{document}
