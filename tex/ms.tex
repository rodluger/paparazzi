\documentclass[modern]{aastex62}

% Load the corTeX style definitions
\input{cortex}

% Bibliography stuff
\bibliographystyle{aasjournal}

% Begin!
\begin{document}

% Title
\title{Doppler Imaging Fun}

% Author list
\author[0000-0002-0296-3826]{Rodrigo Luger}
\email{rluger@flatironinstitute.org}
\affil{Center~for~Computational~Astrophysics, Flatiron~Institute, New~York, NY}
%
\author{Megan Bedell}
\affil{Center~for~Computational~Astrophysics, Flatiron~Institute, New~York, NY}
%
\author{Probably David W. Hogg}
\affil{Center~for~Computational~Astrophysics, Flatiron~Institute, New~York, NY}

%
\section{Introduction}
Check out \citet{Luger2019} and \citet{Bedell2019} and stuff.

%
\section{Uniform surfaces}
\label{sec:uniform surfaces}

Let $I(\xi, x, y)$ be the Doppler-shifted intensity observed at log wavelength $\xi \equiv \ln\lambda$ at
sky-projected Cartesian position $x, y$ on the surface of the star. We may express it as
%
\begin{align}
    \label{eq:IntensityUnif}
    I(\xi, x, y) &= I_0(\xi_0)
\end{align}
%
where $I_0$ is the (spatially uniform) spectrum in the original, unshifted frame
and 
%
\begin{align}
    \xi_0 &= \xi - \alpha(x, y)
\end{align}
%
is the log wavelength in the unshifted frame. The quantity $\alpha$ is
computed from the formula for the relativistic Doppler shift and is equal to
%
\begin{align}
    \alpha(x, y) &\equiv \frac{1}{2}\ln\left( \frac{1 - \nicefrac{v}{c}}{1 + \nicefrac{v}{c}} \right)
\end{align}
%
where $v = v(x, y)$ is the radial velocity at a point on 
the surface of the star and $c$ is the speed of light.

If we Taylor expand Equation~(\ref{eq:IntensityUnif}) about $\alpha = 0$, we obtain
%
\begin{align}
    \label{eq:TaylorUnifExplicit}
    I(\xi, x, y) 
        &=
        I_0(\xi_0) \Bigg|_{\alpha=0}
        + 
        \frac{\mathrm{d}I_0(\xi_0)}{\mathrm{d}\alpha} \Bigg|_{\alpha=0} \Delta\alpha(x, y)
        + 
        \frac{1}{2}\frac{\mathrm{d}^2I_0(\xi_0)}{\mathrm{d}\alpha^2} \Bigg|_{\alpha=0} \Delta\alpha(x, y)^2
        +
        ... 
\end{align}
%
The derivatives of the spectrum $I_0(\xi_0)$ with respect to
$\alpha$ are computed by repeated application of the chain rule:
%
\begin{align}
    \frac{\mathrm{d}^nI_0(\xi_0)}{\mathrm{d}\alpha^n} &=
    (-1)^n\dfrac{\mathrm{d}^nI_0(\xi_0)}{\mathrm{d}\xi_0^n}
\end{align}
%
Given this result, and noting that $\xi_0 = \xi$ when $\alpha = 0$,
we may re-write Equation~(\ref{eq:TaylorUnifExplicit}) as
%
\begin{align}
    \label{eq:TaylorUnifSum}
    I(\xi, x, y) 
        &=
        I_0(\xi)
        +
        \sum_{n=1}^\infty
            \frac{(-1)^n}{n!}
            \frac{\mathrm{d}^nI_0(\xi)}{\mathrm{d}\xi^n}
            \Delta\alpha(x, y)^n
\end{align}



% Bibliography
\bibliography{bib}

\end{document}
