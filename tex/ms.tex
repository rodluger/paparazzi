% Define document class & import showyourwork
\documentclass[modern]{aastex631}

% Begin!
\begin{document}

% Title
\title{A Closed-Form Solution to the Doppler Imaging Problem}

% Author list
\author[0000-0002-0296-3826]{Rodrigo Luger}
\email{rluger@flatironinstitute.org}
\affil{Center~for~Computational~Astrophysics, Flatiron~Institute, New~York, NY}
%
\author{Megan Bedell}
\affil{Center~for~Computational~Astrophysics, Flatiron~Institute, New~York, NY}
%
\author{Daniel Foreman-Mackey}
\affil{Center~for~Computational~Astrophysics, Flatiron~Institute, New~York, NY}
%
\author{David W. Hogg}
\affil{Center~for~Computational~Astrophysics, Flatiron~Institute, New~York, NY}

\begin{abstract}
    We derive a closed form, analytic solution to the problem of Doppler imaging of stellar surfaces in the limit of negligible differential rotation and convective blueshift.
    The model for the observed spectrum is linear in the coefficients of the spherical harmonic expansion of the specific intensity distribution on the surface and, in certain limits, the posterior over surface maps has a closed form, analytic solution that is computationally trivial to evaluate.
    The model is also linear in the local (rest frame) stellar spectrum, which may itself be spatially variable.
    This allows one to perform Doppler imaging without knowledge of the local stellar spectrum and therefore works on blended lines or regions of the spectrum where line formation mechanisms are not well understood.
    Finally, the model is fast, differentiable, and allows one to calculate uncertainties on the inferred surface map.
\end{abstract}

% Main body
\section{Introduction}
The paper is organized as follows: in \S\ref{sec:the_problem}~and~\S\ref{sec:the_solution} we introduce the math behind the Doppler imaging problem and derive a closed form expression for the model. 
In \S\ref{sec:linear},~\S\ref{sec:inverse},~and~\S\ref{sec:bellswhistles} we demonstrate how to re-express the model as a linear operation on the input spectrum and stellar map and derive closed form expressions for the two conditioned on the data and priors.
In \S\ref{sec:spotstar} we apply our techniques to a mock problem. 
We further discuss and summarize our results in \S\ref{sec:discussion}~and~\S\ref{sec:conclusions}. 
Finally, auxiliary derivations are presented in the Appendix.


\end{document}