\documentclass[modern]{aastex62}

% Load the corTeX style definitions
% All the packages
\usepackage{url}
\usepackage{amsmath}
\usepackage{mathtools}
\usepackage{amssymb}
\usepackage{natbib}
\usepackage{graphicx}
\usepackage{calc}
\usepackage{etoolbox}
\usepackage{xspace}
\usepackage[T1]{fontenc} % https://tex.stackexchange.com/a/166791
\usepackage{textcomp}
\usepackage{ifxetex}
\ifxetex
\usepackage{fontspec}
\defaultfontfeatures{Extension = .otf}
\fi
\usepackage{fontawesome}
\usepackage{listings}
\usepackage{nicefrac}
\usepackage[bb=boondox]{mathalfa}


% Shorthand for this paper
\newcommand{\Python}{\textsf{Python}\xspace}
\newcommand{\cpp}{\textsf{C}++\xspace}
\newcommand{\bvec}[1]{{\ensuremath{\mathbf{#1}}}}
\newcommand{\xxx}[1]{{\color{red}#1}}
\DeclarePairedDelimiter\floor{\lfloor}{\rfloor}
\DeclarePairedDelimiter\ceil{\lceil}{\rceil}
\newcommand{\imag}{{\ensuremath{\mathbb{i}}}}

% References to text content
\newcommand{\documentname}{\textsl{article}}
\newcommand{\figureref}[1]{\ref{fig:#1}}
\newcommand{\Figure}[1]{Figure~\figureref{#1}}
\newcommand{\figurelabel}[1]{\label{fig:#1}}
\renewcommand{\eqref}[1]{\ref{eq:#1}}
\newcommand{\Eq}[1]{Equation~(\eqref{#1})}
\newcommand{\eq}[1]{\Eq{#1}}
\newcommand{\eqalt}[1]{Equation~\eqref{#1}}

% Add code, proof, and animation hyperlinks
\definecolor{linkcolor}{rgb}{0.1216,0.4667,0.7059}
\newcommand{\codeicon}{{\color{linkcolor}\faFileCodeO}}
\newcommand{\prooficon}{{\color{linkcolor}\faPencilSquareO}}
% !TeX root = ./ms.tex
\newcommand{\codelink}[1]{\href{https://github.com/user/repo/blob/076a0d29804b1875a480b0fd74a7ea6738368263/tex/figures/#1.py}{\codeicon}\,\,}
\newcommand{\animlink}[1]{\href{https://github.com/user/repo/blob/076a0d29804b1875a480b0fd74a7ea6738368263/tex/figures/#1.gif}{\animicon}\,\,}
\newcommand{\prooflink}[1]{\href{https://github.com/user/repo/blob/076a0d29804b1875a480b0fd74a7ea6738368263/tex/proofs/#1.ipynb}{\raisebox{-0.1em}{\prooficon}}}
\newcommand{\cilink}[1]{\href{https://dev.azure.com/user/repo/_build}{#1}}


% Define a proof environment for open source equation proofs
\newtagform{eqtag}[]{(}{)}
\newcommand{\currentlabel}{None}
\newenvironment{proof}[1]{%
\ifstrempty{#1}{%
\renewtagform{eqtag}[]{\raisebox{-0.1em}{{\color{red}\faPencilSquareO}}\,(}{)}%
}{%
\renewtagform{eqtag}[]{\prooflink{#1}\,(}{)}%
}%
\usetagform{eqtag}%
\renewcommand{\currentlabel}{#1}
\align%
}{%
\endalign%
\renewtagform{eqtag}[]{(}{)}%
\usetagform{eqtag}%
\message{<<<\currentlabel: \theequation>>>}%
}

% Define the `oscaption` command for open source figure captions
\newcommand{\oscaption}[2]{\caption{#2 \codelink{#1}}}

% Code examples
\definecolor{codegreen}{rgb}{0,0.6,0}
\definecolor{codegray}{rgb}{0.5,0.5,0.5}
\definecolor{codepurple}{rgb}{0.58,0,0.82}
\definecolor{backcolour}{rgb}{0.95,0.95,0.95}
\lstdefinestyle{mystyle}{
    backgroundcolor=\color{backcolour},
    commentstyle=\color{codegreen},
    keywordstyle=\color{magenta},
    numberstyle=\tiny\color{codegray},
    stringstyle=\color{codepurple},
    basicstyle=\small\ttfamily,
    breakatwhitespace=false,
    breaklines=true,
    captionpos=b,
    keepspaces=true,
    numbers=left,
    numbersep=5pt,
    showspaces=false,
    showstringspaces=false,
    showtabs=false,
    tabsize=2,
    aboveskip=1em,
    belowskip=1em,
    keywords=[2]{map},
    keywordstyle=[2]{\color{black!80!black}},
    upquote=true
}
\lstset{style=mystyle}

% Typography obsessions
\setlength{\parindent}{3.0ex}
\renewcommand\quad{\hskip\fontdimen3\font}


% Bibliography stuff
\bibliographystyle{aasjournal}

% Begin!
\begin{document}

% Title
\title{Doppler Imaging Fun}

% Author list
\author[0000-0002-0296-3826]{Rodrigo Luger}
\email{rluger@flatironinstitute.org}
\affil{Center~for~Computational~Astrophysics, Flatiron~Institute, New~York, NY}
%
\author{Megan Bedell}
\affil{Center~for~Computational~Astrophysics, Flatiron~Institute, New~York, NY}
%
\author{Probably David W. Hogg}
\affil{Center~for~Computational~Astrophysics, Flatiron~Institute, New~York, NY}

%
\section{Introduction}
Check out \citet{Luger2019} and \citet{Bedell2019} and stuff.

%
\section{Uniform surfaces}
\label{sec:uniform surfaces}

Let $I(\lambda, x, y)$ be the Doppler-shifted intensity observed at wavelength $\lambda$ at
sky-projected Cartesian position $x, y$ on the surface of the star. We may express it as
%
\begin{align}
    \label{eq:IntensityUnif}
    I(\lambda, x, y) &= I_0\Big(\mathcal{D}^{-1}_{\beta(x, y)}(\lambda)\Big)
\end{align}
%
where $I_0$ is the (spatially uniform) spectrum in the original, unshifted frame
and 
%
\begin{align}
    \mathcal{D}^{-1}_{\beta(x, y)}(\lambda) &= \frac{\lambda}{1 + \beta(x, y)} \nonumber \\
                                            &\equiv \lambda_0
\end{align}
%
is the non-relativistic \emph{inverse} Doppler operator, which takes as input the shifted (observed)
wavelength $\lambda$ and returns the unshifted wavelength $\lambda_0$.
The quantity
%
\begin{align}
    \beta(x, y) &= \frac{v(x, y)}{c}
\end{align}
%
is the ratio of the radial velocity $v(x, y)$ at a point on the
surface to the speed of light.

If we Taylor expand Equation~(\ref{eq:IntensityUnif}) about $\beta = 0$, we obtain
%
\begin{align}
    \label{eq:TaylorUnifExplicit}
    I(\lambda, x, y) 
        &=
        I_0(\lambda_0) \Bigg|_{\beta=0}
        + 
        \frac{\mathrm{d}I_0(\lambda_0)}{\mathrm{d}\beta} \Bigg|_{\beta=0} \Delta\beta(x, y)
        + 
        \frac{\mathrm{d}^2I_0(\lambda_0)}{\mathrm{d}\beta^2} \Bigg|_{\beta=0} \Delta\beta(x, y)^2
        +
        ... 
\end{align}
%
The derivatives of the spectrum $I_0(\lambda_0)$ with respect to
$\beta$ are found by application of Fa\`a di Bruno's formula for taking high
order derivatives of the chain rule:
%
\begin{proof}{faa_di_bruno}
    \label{eq:dIdbeta}
    \frac{\mathrm{d}^n I_0(\lambda_0)}{\mathrm{d}\beta^n} \Bigg|_{\beta=0} &=
    \sum_{k=1}^n \frac{\mathrm{d}^k I_0(\lambda_0)}{\mathrm{d}\lambda_0^k} \lambda_0^k P_{nk}
\end{proof}
%
where
%
\begin{proof}{faa_di_bruno}
    \label{eq:Pnk}
    P_{nk} \equiv B_{n, k}\Bigg( \Big\{(-1)^j j! \Big\}_{j=1}^{n - k + 1} \Bigg)
\end{proof}
%
is an integer computed from the incomplete Bell polynomial $B_{n, k}$.
Given this result, and noting that $\lambda_0 = \lambda$ when $\beta = 0$,
we may re-write Equation~(\ref{eq:TaylorUnifExplicit}) as
%
\begin{align}
    \label{eq:TaylorUnifSum}
    I(\lambda, x, y) 
        &=
        I_0(\lambda)
        + 
        \sum_{n=1}^\infty
            \Bigg(
                \sum_{k=1}^n \frac{\mathrm{d}^k I_0(\lambda)}{\mathrm{d}\lambda^k} \lambda^k P_{nk} 
            \Bigg)
            \Bigg(
                \Delta\beta(x, y)^n
            \Bigg)
\end{align}
%
Assuming we only have access to spatially unresolved spectra of the star, the quantity
we observe is the integral of the intensity over the surface of the star, which we
denote $S$:
%
\begin{align}
    \label{eq:TaylorUnifSumIntegral}
    S(\lambda) 
        &=
        \pi I_0(\lambda)
        + 
        \sum_{n=1}^\infty
            \Bigg(
                \sum_{k=1}^n \frac{\mathrm{d}^k I_0(\lambda)}{\mathrm{d}\lambda^k} \lambda^k P_{nk} 
            \Bigg)
            \Bigg(
                \int{\Delta\beta(x, y)^n}\mathrm{d}\Omega
            \Bigg)
\end{align}
%
where $\int\mathrm{d}\Omega$ denotes the surface integral over the visible disk
of the star.

Note, importantly, that the terms in the first set of parentheses depend \emph{only} on the wavelength,
while the terms in the second set of parentheses depend \emph{only} on the position.
This Taylor expansion has therefore allowed us to deconvolve the effects of the Doppler
shift from the actual spectrum of the star.

As an example of an application of this equation, consider the case of a rigidly rotating,
uniform star with equatorial velocity in units of the speed of light equal to $\beta_\mathrm{eq}$. 
It is easy to show that in this case, $\beta(x, y) = \beta_\mathrm{eq} x$. The integrals
in Equation~(\ref{eq:TaylorUnifSumIntegral}) are trivial to compute:
%
\begin{align}
    \int{\Delta\beta(x, y)^n}\mathrm{d}\Omega = 
        \begin{cases} 
            0 & \quad\quad\quad\quad\quad n \, \mathrm{odd} \\
            \beta_\mathrm{eq}\sqrt{\pi}\frac{\Gamma\big(\frac{n}{2} + \frac{1}{2}\big)}{\Gamma\big(\frac{n}{2} + 2\big)} & \quad\quad\quad\quad\quad n \, \mathrm{even}
        \end{cases}
\end{align}
%
so the expression for the spectrum becomes
\begin{align}
    \label{eq:LineBroadenedSpectrum}
    S(\lambda) 
        &=
        \pi I_0(\lambda)
        + 
        \sqrt{\pi}
        \sum_{n\,\mathrm{even}}
            \beta_\mathrm{eq}^n
            \frac{\Gamma\big(\frac{n}{2} + \frac{1}{2}\big)}{\Gamma\big(\frac{n}{2} + 2\big)}
            \sum_{k=1}^n 
                \frac{\mathrm{d}^k I_0(\lambda)}{\mathrm{d}\lambda^k} \lambda^k P_{nk} 
\end{align}
%
Equation~(\ref{eq:LineBroadenedSpectrum}) is the analytic expression for line broadening
due to a uniform rotating star.


% Bibliography
\bibliography{bib}

\end{document}
